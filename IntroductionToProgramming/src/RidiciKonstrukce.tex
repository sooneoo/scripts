\chap {Řídící konstrukce}
{\bf Řídicí konstrukce} jsou základní nástroje jazyka C, které slouží k popisu algoritmu. Zatímco proměnné představují data, řídicí konstrukce určují logiku, jakou se s těmito daty bude pracovat.

\sec {Blok kódu a rozsah platnosti}
{\bf Blok kódu}, vymezený složenými závorkami $\{$ a $\}$, představuje v programování základní realizaci řídicí konstrukce sekvence. Jde o logický celek, ve kterém se instrukce vykonávají postupně. Tento strukturální prvek zároveň definuje ztv {\bf rozsah platnosti} (scope), který ovlivňuje životnost proměnných, tedy úsek kódu ve kterém lze na danou proměnnou odkazovat. 

Pokud se vytvoří proměnná {\it number} uvnitř daného bloku

\ttline=0
\begtt
{
int number = 42;
...
}
\endtt
\ttline=-1

K této proměnné lze pak přistupovat pouze uvnitř tohoto bloku ve kterém vznikla, za hranicemi bloku proměnná {\bf number} již zaniká a nelze s ní již dál pracovat!

\secc Vnořený blok kódu
Sekvence příkazů uvnitř bloku tvoří uzavřený kontext. Pokud se do sekvence vloží další dvojice složených závorek, vytvoří se tak {\bf vnořený blok}. Tento mechanismus umožňuje systému efektivně členit sekvence kódu se kterými lze pracovat jako s celkem.

\sec Větvení programu


\secc Konstrukce if-else


\secc Konstrukce switch-case


\sec Iterace programu


\secc Smyčka for


\secc Smyčka while


\sec Rekurze


