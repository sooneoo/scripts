\chap {Komentáře}

Psaní kódu je kreativní proces, a stejně jako každá tvůrčí práce, může se stát s postupem času složitější. Někdy se i ten nejsrozumitelnější kód stane obtížným na pochopení, zvláště když na něm pracuje více lidí. Z tohoto důvodu každý programovací jazyk obsahuje konstrukci, která umožňuje vkládat informativní poznámky do zdrojového kódu, díky kterým je možné vytvořit srozumitelnější a čitelnější kód.


\sec {Dva druhy komentářů}

Komentáře jsou části kódu, které překladač zcela ignoruje. Slouží jako poznámky a vysvětlivky pro udržení srozumitelnosti kódu, nebo pro tvorbu komplexní dokumentace. V jazyce C rozlišujeme dva typy: {\bf řádkové komentáře} a {\bf blokové komentáře}

\secc {Řádkový komentář}

Řádkový komentář se vytváří pomocí sekvence dvou lomítek a umožňuje vytvářet komentář v rámci jednoho řádku. Důležité je že řádkový komentář je definovaný od zahajovací sekvence lomítek až do konce řádku:

\ttline=0
\begtt
// this is line comment
\endtt
\ttline=-1

\secc {Blokový komentář}
Blokový komentář se vytváří pomocí uvozovací sekvence {\it /*} a ukončovací sekvence {\it */} a cokoliv definované uvnitř je překladačem považováno za komentář nehledě na odřádkování.

\ttline=0
\begtt
/*
this is
block comment
defined on multiple lines
*/
\endtt
\ttline=-1

\sec {Správné použití komentářů}
Komentáře jsou mocný nástroj, který by měl každý programátor používat již od začátku. Přispívají k efektivitě a kvalitě kódu, ovšem pokud se používají nesprávně, mohou naopak působit matoucím dojmem a snižovat čitelnost výsledného kódu. 

Komentáře by se měly do zdrojových kódů zapisovat ideálně v anglickém jazyce, protože díky tomu jim bude rozumět i někdo kdo nemluví česky. Rozhodně nemíchat české a anglické komentáře!

Nejdůležitější pravidlo zní: {\it Nekomentuj, co kód dělá, ale proč to dělá. Pokud je kód sám o sobě srozumitelný, komentář není potřeba}. 

Vždy je snahou spíš psát takový kód, který je srozumitelný sám o sobě a nevyžaduje komentáře a komentovat pouze to co kód sám nedokáže říct. 

\break 
\sec {Příklad použití komentářů}

\ttline=0
\begtt
/*
Overview of the source code:
HelloWorld project demonstrating the basic features of a
programming language
*/

#include <stdio.h>

int main(void) {
    printf("Hello World\n"); // printing message to console
    return 0;
} 
\endtt
\ttline=-1



