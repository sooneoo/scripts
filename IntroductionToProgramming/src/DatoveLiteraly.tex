\chap Konstanty a datové literály
V jazyce C je běžné do programu přímo vkládat pevné hodnoty, jako jsou {\bf čísla}, {\bf znaky} nebo {\bf textové řetězce}. Tyto doslovně zapsané hodnoty se nazývají {\bf literály}.

Literály se používají k inicializaci proměnných nebo k definování pevných pravidel v logice programu. Příkladem může být porovnání výsledku výpočtu s předem danou maximální nebo minimální hodnotou. 

\sec Druhy datových literálů
Při programování je klíčové rozlišovat různé druhy literálů, protože každý z nich má svůj datový typ, který určuje, jak s ním má počítač správně pracovat. V jazyce C se rozlišují čtyři základní typy: 
\begitems
* {\bf Celočíselné literály}
* {\bf Desetinné literály}
* {\bf Znakové literály}
* {\bf Řetězcové literály}
\enditems

Znakové literály jsou svázány s datovým typem {\bf char} a představují {\bf jeden znak} v jednoduchých uvozovkách.

\ttline=0
\begtt
char single_character = 'a';
\endtt
\ttline=-1

Celočíselné literály představují celá čísla a nejčastěji se vážou na datový typ {\bf int}. Protože jsou ale znaky v počítači rovněž reprezentovány pomocí čísel, je možné číselné literály reprezentovat také pomocí datového typu {\bf char}.

\ttline=0
\begtt
int number = 42;
char small_number = 27;
\endtt
\ttline=-1

Desetinné literály představují {\bf číselné hodnoty s desetinným místem}. Tyto hodnoty se v jazyce C zapisují s {\bf desetinnou tečkou}. Desetinné literály se vážou s datovým typem {\bf double}  nebo {\bf float}.

\ttline=0
\begtt
float decimal_number = 3.14;
float long_decimal_number = 3.141592;
\endtt
\ttline=-1

Řetězcové literály reprezentují sekvenci znaků tvořící text, který je uzavřen v dvojitých uvozovkách. Technicky se jedná o ukazatel na pole znaků.

\ttline=0
\begtt
char * string = "Hello, World!";
\endtt
\ttline=-1



