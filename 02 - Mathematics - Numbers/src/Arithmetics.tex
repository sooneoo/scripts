\chap Aritmetika

Aritmetika je nejstarší a nejzákladnější odvětví matematiky, které se zabývá počítáním a základními operacemi s čísly. Tyto operace umožňují řešit problémy od jednoduchého sčítání až po komplexní vědecké výpočty. Aritmetika je jazyk, který  pomáhá provádět výpočty a pochopit, jak se čísla chovají.

\sec Čtyři základní operace
Aritmetika je založena na čtyřech hlavních operacích: sčítání, odčítání, násobení a dělení.

\begitems 
* {\bf Sčítání (+)} - Operace, která spojuje dvě nebo více hodnot (sčítance) do jedné hodnoty (součet).
* {\bf Odčítání (-)} - Operace, která určuje rozdíl mezi dvěma čísly. Jedná se o opačnou operaci ke sčítání.
* {\bf Násobení ($\cdot$)} - Operace opakovaného sčítání.
* {\bf Dělení ($\div$)} - Operace opakovaného odčítání. Jedná se o opačnou operaci k násobení. 
\enditems

\sec Operace sčítání
{\bf Operace součet} neboli {\bf sčítání} je jednou ze čtyř základních aritmetických operací. Jejím účelem je spojit dvě nebo více čísel {\bf nazývaných sčítanci} do jedné výsledné hodnoty, která se označuje jako {\bf součet}. Operace součet se zapisuje pomocí symbolu "+": 

$$ A + B = C $$

Sčítání se řídí několika důležitými vlastnostmi, které jsou klíčové pro správné a efektivní počítání. Tyto vlastnosti platí pro většinu číselných oborů, jako jsou přirozená, celá nebo reálná čísla.

\begitems
* {\bf Komutativita}: Na pořadí sčítanců nezáleží. Bez ohledu na to, v jakém pořadí čísla sečtete, výsledek zůstane stejný: $a + b = b + c$ 
* {\bf Asociativita}: Při sčítání tří nebo více čísel nezáleží na tom, jak jsou sčítanci uzávorkováni: $ (a + b) + c = a + (b + c) $
* {\bf Existence neutrálního prvku}: Číslo nula je neutrálním prvkem pro sčítání. To znamená, že přičtením nuly k libovolnému číslu se toto číslo nezmění $a + 0 = a$
\enditems

\sec Operace násobení
Operace násobení neboli součin je zkrácený zápis pro opakované sčítání. Pomáhá nám rychle vypočítat součet, pokud přidáváme stejné číslo vícekrát. Násobená čísla, se nazývají {\bf činitelé} a výsledek se nazývá {\bf součin}. Operace násobení se zapisuje pomocí symbolu "$\cdot$":

$$ A \cdot B = C $$

Podobně jako sčítání, i násobení má několik klíčových vlastností, které určují, jak se chová.

\begitems
* {\bf Komutativita}: Na pořadí činitelů nezáleží. Součin zůstane stejný bez ohledu na to, v jakém pořadí je vynásobíte: $ a \cdot b = b \cdot a $
* {\bf Asociativita}: Při násobení tří nebo více čísel nezáleží na tom, jak jsou činitelé uzávorkováni: $(a \cdot b) \cdot c = a \cdot (b \cdot c) $
* {\bf Existence neutrálního prvku}: Číslo jedna je neutrálním prvkem pro násobení. To znamená, že vynásobením libovolného čísla jedničkou se toto číslo nezmění: $a \cdot 1 = a$
* {\bf Distributivita}: Tato vlastnost je jedinečná, protože spojuje násobení se sčítáním. Násobení je "distributivní" vůči sčítání, což znamená, že můžete roznásobit číslo do závorky: $a \cdot (b + c) = a \cdot b + \cdot c$
\enditems

\sec Operace odčítání
Odčítání je aritmetická operace, která je inverzní ke sčítání. Používá se k určení rozdílu mezi dvěma čísly. Zatímco sčítání je o přidávání, odčítání je o odebírání nebo hledání chybějící části. Číslo, od kterého se odčítá, se nazývá {\bf menšenec}, číslo, které se odčítá, se nazývá {\bf menšitel} a výsledek je {\bf rozdíl}. Operace odčítání se zapisuje pomocí symbolu "-":

$$ A - B = C $$

Na rozdíl od sčítání a násobení nemá odčítání stejné vlastnosti a je méně flexibilní.

\begitems
* {\bf Není komutativní}: Na pořadí čísel při odčítání záleží. Změna pořadí vede k odlišnému: $a - b \neq b - a$
* {\bf Není asociativní}: Při odčítání tří nebo více čísel záleží na tom, jak jsou uzávorkovány: $(a - b) - c \neq a - (b - c)$
* {\bf Vztah se sčítáním}: Odčítání je možné převést na sčítání přičtením opačného čísla $a - b = a + (-b)$
\enditems

\sec Operace dělení
Dělení je aritmetická operace, která je inverzní k násobení. Používá se k určení, kolikrát se jedno číslo, tzv. dělitel, vejde do druhého čísla, dělence.  Výsledkem dělení je podíl. Dělení si můžete představit jako proces rozdělování celku na stejné části. Operace dělení se zapisuje pomocí symbolu "$\div$":

$$ A \div B = C $$

Dělení se svými vlastnostmi podobá odčítání, je méně flexibilní než sčítání a násobení.

\begitems
* {\bf Není komutativní}: Na pořadí čísel při dělení záleží. Změna pořadí vede k odlišnému výsledku: $a \div b \neq b \div a$
* {\bf Není asociativní}: Při dělení tří nebo více čísel záleží na tom, jak jsou uzávorkovány $a \div (b \div c) \neq (a \div b) \div c$
* {\bf Vztah s násobením}: Dělení je možné převést na násobení převrácenou hodnotou $a \div b = a \cdot {1 \over b} $
\enditems

Speciálním případem dělení je dělení nulou. Dělení nulou není v matematice definováno. Neexistuje totiž číslo, které by po vynásobení nulou dalo jiný výsledek než nulu. Proto se pro žádné číslo {\it a} (kromě nuly samotné) nedá vypočítat $ a \div 0$. Z tohoto důvodu je nutné se tomuto typu operace vyhnout.

\sec Priorita operací
Při řešení matematických výrazů, které obsahují více než jednu operaci, je nezbytné dodržovat přesné pořadí. Toto pořadí se nazývá {\bf priorita operací} a jedná se o dohodnutou konvenci, nikoli o matematický zákon. Byla zavedena, aby se předešlo nejednoznačnostem a aby každý výraz měl vždy pouze jeden správný výsledek.

\begitems \style n
* {\bf Závorky}: Závorky mají nejvyšší prioritu, protože jejich účelem je úmyslně přepsat dohodnutá pravidla. Jejich existence ve výrazu je signálem, že se daná část musí vyřešit jako první, bez ohledu na to, jaké operace jsou uvnitř.
* {\bf Mocniny}: Mocniny a odmocniny mají přednost před násobením a dělením, protože představují opakované násobení.
* {\bf Násobení a dělení}: Násobení má přednost před sčítáním, protože se jedná o opakované sčítání. Dělení má stejnou prioritu jako násobení, protože je jeho inverzní operací. Dělení je v podstatě násobení převrácenou hodnotou. Vzhledem k jejich úzkému vztahu se provádějí ve stejné úrovni priority.
* {\bf Sčítání a odčítání}: Sčítání a odčítání mají nejnižší prioritu, protože jsou to nejzákladnější aritmetické operace. Podobně jako u násobení a dělení, odčítání je inverzní operace ke sčítání a proto mají stejnou prioritu.
\enditems


\sec Sumace
{\bf Sumace} je operace, která umožňuje zkráceným způsobem zapsat součet řady čísel. Místo dlouhého vypisování všech sčítanců se využívá  speciální zápis. Tato operace je nepostradatelná v mnoha oblastech matematiky, jako je statistika, finančnictví nebo teoretická fyzika.

Pro sumaci používáme symbol $\sum$, což je velké řecké písmeno sigma. Zápis vypadá takto: 
$$\sum_{i=m}^{n} a_i $$.


\begitems
* $\sum$ - Symbol pro sumaci
* $i$ - {\bf Sčítací index} nebo {\bf index sumace}. Je to proměnná, která se postupně mění.
* $m$ - {\bf Dolní mez} sumace. Je to hodnota, kterou index {\it i} nabývá na začátku.
* $n$ - {\bf Horní mez} sumace. Je to hodnota, kterou index {\it i} nabývá na konci. 
* $a_i$ - {\bf Sčítaný člen}. Je to výraz, jehož hodnoty sčítáme. Může to být konstanta nebo výraz závislý na indexu {\it i}.
\enditems

Výraz $\sum_{i=m}^{n} a_i$ znamená, že se sečtou hodnoty výrazu $a_i$ pro všechna celá čísla {\it i} od dolní meze {\it m} až do horní meze {\it n}, včetně obou mezí. Například:

$$ \sum_{i=1}^{5} i = 1 + 2 + 3 + 4 + 5 = 15 $$

\sec Produkt

{\bf Produkt} je aritmetická operace, která slouží ke zkrácenému zápisu součinu řady čísel. Stejně jako sumace, i tato operace je velmi užitečná v mnoha oblastech matematiky a vědy, protože umožňuje efektivně pracovat s dlouhými řadami čísel.

Pro produkt se používá symbol $\prod$, což je velké řecké písmeno {\bf pí}. Zápis je velmi podobný zápisu pro sumaci:

$$ \prod_{i = m} ^ {n} a_i $$


\begitems
* $\sum$ - Symbol pro produkt
* $i$ - {\bf Index součinu}. Je to proměnná, která se postupně mění.
* $m$ - {\bf Dolní mez} součinu. Je to hodnota, od které index {\it i} začíná. 
* $n$ - {\bf Horní mez} součinu. Je to hodnota, u které index {\it i} končí.
* $a_i$ - {\bf Člen součinu}. Je to výraz, jehož hodnoty se násobí.
\enditems

Výraz $\prod_{i=m}^{n} a_i$ znamená, že se násobí hodnoty výrazu $a_i$ pro všechna celá čísla $i$ od dolní meze $m$ až do horní meze $n$, včetně obou mezí. Například:

$$ \prod_{i=1}^{5} i = 1 \cdot 2 \cdot 3 \cdot 4 \cdot 5 = 24 $$

Operace produktu je klíčovým nástrojem, který umožňuje pracovat se složitými výpočty, aniž by bylo nutné vypisovat dlouhé řady násobení. Je to přímé zobecnění základní operace násobení, stejně jako sumace zobecňuje sčítání.



