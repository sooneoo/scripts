\chap Úvod

Matematika je jazyk, který umožňojue popisovat svět kolem. Základním stavebním kamenem tohoto jazyka jsou čísla. Čísla však nejsou jen jednoduché symboly. Představují komplexní systém, který se vyvíjel po tisíce let, aby vyhověl rostoucím potřebám lidstva. 

\sec Číselný obor
{\bf Číselným oborem} nebo {\bf číselnou množinou} se rozumí množina, na které jsou definovány bez omezení základní početní operace (sčítání, odčítání, …) - číselný obor je vzhledem k těmto operacím uzavřený. {\bf Uzavřenost (úplnost)} číselného oboru vzhledem k početní operaci znamená, že výsledkem početní operace mezi dvěma libovolnými prvky z příslušné číselné množiny je opět číslo, které také patří do této číselné množiny.
Množina všech čísel určitého druhu, ve které jsou definovány bez omezení operace sčítání a násobení se nazývá číselný obor.

\begitems
* \N \hskip 3px - slouží k počítání a vyjádření množství fyzických objektů.
* \Z \hskip 3px - umožňuje vyjádřit dluhy a směr na ose (např. teplota), přesahuje tak sféru fyzických objektů.
* \Q \hskip 3px - je nezbytné pro měření, dělení a vyjádření necelých částí celku. 
* \R \hskip 3px - dává kompletní a souvislý systém pro práci s fyzikálními veličinami, jako je čas, vzdálenost a hmotnost.
\enditems

Přičemž platí: 

$$ \N \subset \N_0 \subset \Z \subset \Q \subset \R \subset \C $$

Studium číselných oborů je tedy zásadní, protože dává jasný přehled o tom, jaká čísla lze v dané situaci použít a jaké operace s nimi lze bez problémů provádět.

\sec Rozvoj číselných oborů

Historie čísel je spjata s praktickými potřebami. Prvními čísly, která lidé používali, byla přirozená čísla ({\N}). Sloužila k počítání a určování množství, ať už šlo o ovce ve stádě nebo dny v kalendáři. Ale brzy se objevily složitější problémy. Co když je třeba vyjádřit dluh, nebo teplotu pod bodem mrazu? Proto lidé zavedli záporná čísla a nulu, čímž vznikl obor celých čísel (\Z).

S rozvojem obchodu, stavitelství a vědy bylo nezbytné dělit a měřit s větší přesností. Jak vyjádřit polovinu chleba nebo čtvrtinu pozemku? Pro tento účel byly zavedeny {\bf zlomky} a {\bf desetinná čísla}, které tvoří obor racionálních čísel (\Q).

Avšak ani racionální čísla nebyla dostatečná. Řečtí matematici objevili, že existují délky, které nelze vyjádřit jako zlomek. Jedním z prvních a nejznámějších příkladů je délka přepony rovnoramenného pravoúhlého trojúhelníku, kde obě odvěsny mají délku 1. Délka přepony je $\sqrt {2}$, což je číslo, které nelze vyjádřit jako poměr dvou celých čísel. To vedlo ke vzniku iracionálních čísel (\I). Sjednocením racionálních a iracionálních čísel vznikl obor reálných čísel (\R), který pokrývá celou číselnou osu bez jakýchkoli "mezer".


\sec Číslo 
Číslo je matematická entita reprezentující určitou hodnotu (kvantitu). Jedná o se způsob vyjádření a zaznamenání přesně definovaného množství. Čísla jsou zapisována pomocí znaků, kterým se říká číslice. Pomocí správné kombinace číslic je možné zaznamenat (zapsat) libovolnou číselnou hodnotu. S čísly lze vykonávat určité operace ve kterých mají tato čísla různé vlastnosti.

\sec Obor přirozených čísel
Za přirozené číslo se pokládá každé kladné celé číslo $(1:\infty)$ . Množina přirozených čísel se označuje velkým písmenem \N od slova Natural.

V některých případech je třeba zahrnout do přirozených čísel také číslo nula, například počet prvků prázdné množiny. Proto množina všech přirozených čísel rozšířenou o nulu se označuje jako $ \N_0 = \N \subset {0}$.

Přirozená čísla se využívají ve dvou významech:
\begitems 
* Přirozenými čísly se vyjadřují počty prvků množin - Kolik?
* Přirozenými čísly se vyjadřuje pořadí prvků při jejich uspořádání - Kolikátý?
\enditems


