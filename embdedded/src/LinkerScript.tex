\chap Linker script
Linker skript je textový soubor, který řídí, jak linker (program, který spojuje zkompilovaný kód a knihovny do spustitelného souboru) uspořádá různé části (sekce) výsledného programu v paměti mikrokontroléru. V embedded programování je to klíčový nástroj, protože vám umožňuje přesně definovat, kam se kód a data uloží, což je nezbytné pro efektivní využití omezené paměti.

\sec Základní koncepty a struktura
Linker skript se typicky skládá ze tří hlavních částí: MEMORY, SECTIONS a ENTRY.

\secc MEMORY
Tato sekce popisuje fyzické paměťové oblasti, které má mikrokontrolér k dispozici, jako je FLASH (pro programový kód a konstanty) a RAM (pro proměnné a zásobník). Pro každou oblast definujete její název, počáteční adresu (ORIGIN) a velikost (LENGTH).

Například:

\begtt
MEMORY
{
  FLASH (rx) : ORIGIN = 0x08000000, LENGTH = 128K
  RAM (rwx) : ORIGIN = 0x20000000, LENGTH = 20K
}
\endtt

V tomto příkladu má FLASH počáteční adresu 0x08000000 a velikost 128K, a RAM začíná na 0x20000000 s velikostí 20K. Atributy jako {\it (rx)} a {\it (rwx)} určují, zda je paměťová oblast čitelná, zapisovatelná nebo spustitelná.


Obecně se příkaz zapisuje takto: 
\begtt
MEMORY
  {
    name [(attr)] : ORIGIN = origin, LENGTH = len
    …
  }
\endtt

Popisek "name" odkazuje na název regionu používaný v linker scriptu. Nastavení "attr" je volitený seznam atributů, které nastavují vlastnosti daného sektoru. Tyto parametry mohou být:
\begitems
* 'r' - oddíl pouze pro čtení, například pro FLASH kam se zapisuje seznam instrukcí tvořící program.
* 'w' - oddíl pro čtení i zápis, například oddíl RAM, který uchovává data, která vznikají za běhu programu
* 'x' - oddíl může obsahovat instrukce programu, které mohou být vykonávány procesorem
* 'a' - alokovatelná sekce
* 'i' - inicializovaná sekce
* '!' - obrátí platnost každé atributu, který následuje
\enditems

\secc SECTIONS
Toto je nejdůležitější část. Popisuje, jak linker přebírá vstupní sekce z objektových souborů (zkompilovaný kód) a spojí je do výstupních sekcí ve výsledném spustitelném souboru. Každý objektový soubor vytvořený kompilátorem obsahuje své vlastní sekce (například .text pro kód, .data pro inicializovaná data a .bss pro neinicializovaná data). Funkce linkeru je sloučit všechny tyto vstupní sekce ze všech objektových souborů do jedné jediné výstupní sekce ve finálním spustitelném souboru. Linker skript mu pak k tomu dává přesné pokyny jak to má udělat.

Standardní sekce zahrnují:

\begitems
* {\bf .text}: Obsahuje kód programu. Typicky se umisťuje do FLASH paměti, protože se nemusí měnit za běhu, ale data se musejí uchovávat i po vypnutí napájení.
* {\bf .rodata}: Obsahuje data pouze pro čtení (např. konstanty, textové řetězce). Také se umisťuje do FLASH.
* {\bf .data}: Obsahuje inicializované globální proměnné. Tyto proměnné se sice za běhu mění a jsou proto v RAM, ale jejich počáteční hodnoty musí být uloženy ve FLASH paměti, aby se po startu mikrokontroléru mohly zkopírovat.
* {\bf .bss}: Obsahuje neinicializované globální proměnné. Na začátku programu se tato oblast vynuluje. Tyto proměnné existují pouze v RAM a nezabírají místo ve FLASH.
\enditems

\secc ENTRY
Tento příkaz definuje vstupní bod do programu, tedy první instrukci, která se má provést po spuštění mikrokontroléru.
