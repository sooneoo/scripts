\chap Paměť počítače

\sec Základní Metriky Paměti a Přenosu Dat
V počítačových systémech je klíčové porozumět tomu, jak se data ukládají a přenášejí. K tomu slouží standardizované jednotky a metriky.

\secc Jednotky Kapacity Paměti
Základní jednotkou informace v digitálním světě je bit.

\begitems
* {\bf Bit (b)}: Nezákladnější jednotka informace, která může nabývat pouze dvou stavů: 0 nebo 1 (logická nepravda / logická pravda, vypnuto / zapnuto).
* {\bf Byte (B)}: Skupina bitů, historicky nejčastěji 8 bitů, tvořící nejmenší adresovatelnou jednotku v paměti. Jeden byte se používá k reprezentaci jednoho znaku (např. písmene, číslice) v kódování ASCII. 1 Byte=8 bitu
\enditems

Větší jednotky se odvozují pomocí binárních nebo dekadických násobků.

{\bf Dekadické (SI) Předpony (Desítková Soustava)}

Tyto předpony se používají v telekomunikacích, datových přenosech a často (i když ne vždy přesně) pro marketingové kapacity pevných disků (HDD/SSD).

\begitems
* Kilobyte - KB - $10^3$ - 1000
* Megabyte - MB - $10^6$ - 1 000 00
* Gigabyte - GB - $10^9$ - 1 000 000 000
* Terabyte - TB - $10^12$ - 1 000 000 000 000 
\enditems


{\bf Binární Předpony (Dvojková Soustava)}

Tyto předpony jsou používány pro přesné označení kapacity paměti v operačních systémech a při popisu operační paměti (RAM), kde je výhodné pracovat s mocninami čísla 2. Byly standardizovány komisí IEC.

\begitems
* Kibibyte - KiB - $2^10$ - 1024
* Mebibyte - MiB - $2^20$ - 1 048 576
* Gibibyte - GiB - $2^30$ - 1 073 741 824
* Tebibyte - TiB - $2^40$ - 1 099 511 627 776 
\enditems

\secc Metriky Přenosu Dat
Přenosová rychlost (datový tok) udává, jak rychle se data přenášejí z jednoho místa na druhé. Základní jednotkou je {\bf bit za sekundu} (bit per second). Zde se téměř výhradně používají dekadické předpony, i když je řeč o {\bf bitech}, ne {\bf bytech}.

Bit za sekundu (b/s nebo bps): Udává počet přenesených bitů za jednu sekundu.
\begitems
* {\bf Kilobit za sekundu (kb/s)}: 103 bitů/s.
* {\bf Megabit za sekundu (Mb/s)}: 106 bitů/s.
* {\bf Gigabit za sekundu (Gb/s)}: 109 bitů/s.
\enditems

\secc Latence (Odezva)
Zatímco datový tok udává objem dat, latence udává čas.
\begitems
* {\bf Latence}: Doba zpoždění, která uplyne mezi odesláním požadavku a obdržením první odpovědi (nebo mezi začátkem a koncem přenosu).
* {\bf Jednotky}: Nejčastěji se udává v milisekundách (ms).
* {\bf Význam}: Vysoká latence je typická pro satelitní komunikaci nebo vzdálené servery, zatímco nízká latence je kritická pro online hraní, videokonference a rychlé databázové transakce.
\enditems

\secc Propustnost (Throughput) a Šířka Pásma (Bandwidth)

Tyto pojmy jsou často zaměňovány:
\begitems 
* {\bf Šířka Pásma (Bandwidth)}: Teoretická maximální přenosová rychlost, kterou daný kanál nebo rozhraní dokáže podpořit (např. 1 Gb/s Ethernetová linka má šířku pásma 1 Gb/s). Udává se v b/s.
* {\bf Propustnost (Throughput)}: Skutečná, efektivní rychlost přenosu dat dosažená za reálných podmínek (včetně režie, chyb, zpoždění). Propustnost je vždy menší nebo rovna šířce pásma. Udává se v b/s nebo B/s.
\enditems

\sec Paměťový model
Každý běžící proces v operačním systému (OS) pracuje s vlastním, virtuálním adresovým prostorem. Tento prostor dává procesu iluzi, že má k dispozici souvislý a exkluzivní blok paměti, ačkoli ve skutečnosti je paměť fyzicky rozdělená a sdílená mnoha procesy. Virtuální adresový prostor procesu je obvykle organizován do několika logických segmentů, z nichž každý slouží specifickému účelu.

Paměťový model procesu popisuje, jak je paměť organizována a jak s ní procesy pracují. Obecně se skládá z několika hlavních oblastí, které určují, jak se přistupuje k datům a instrukcím během běhu programu. 
Proces běžící na moderních systémech má paměť rozdělenou oblasti:
\begitems
* {\bf Text} - část paměti kam se ukládají instrukce a konstanty programu, Tato oblast paměti je nastavená jako read-only, aby nedošlo k narušení kódu běžícího programu.
* {\bf Data} - část paměti kde jsou uložené globální a statické inicializované proměnné
* {\bf Bss} - část paměti kde jsou uloženy globální a statické neinicializované proměnné (jsou nastaveny na počáteční hodnotu 0)
* {\bf Heap} - část paměti, která slouží k dynamickému přidělování za běhu programu
* {\bf Stack} - část paměti, která slouží k ukládání kontextu volané funkce (lokální proměnné, návratové hodnoty, ukazatel na zásobník) 
\enditems

\sec Hierarchie Paměti a Cache Paměti
Hierarchie paměti je uspořádání paměťových komponent do pyramidové struktury, která optimalizuje systém tak, aby procesor pracoval co nejčastěji s co nejrychlejší pamětí.

\secc Princip Lokality
Úspěch hierarchie je založen na principu lokality:

\begitems
* {\bf Časová Lokalita}: Pokud byla data právě použita, je pravděpodobné, že budou použita znovu brzy.
* {\bf Prostorová Lokalita}: Pokud byla data na adrese {\bf A} použita, je pravděpodobné, že budou brzy použita sousední data (A+1, A+2, atd.).
\enditems

\secc Úrovně Hierarchie
\begitems
* {\bf L0} - {\bf Registry} - přístup <1 cyklus CPU - kapacity většinou bitová šířka procesoru
* {\bf L1, L2, L3} - {\bf Cache} - přístup 1 až 80 cyklů - kapacita KiB až MiB 
* {\bf L4} - {\bf Hlavní paměť RAM} - přístup 100+ cyklů - kapacita GiB
* {\bf L5} - {\bf Sekundární (trvalé) uložiště} - přístup v řádech milionů cyklů - kapacita TiB
\enditems

\secc Cache Paměť (L1, L2, L3)
{\bf Cache} je malá, rychlá SRAM paměť, která slouží jako vyrovnávací paměť mezi CPU a RAM. Paměť cache (mezipaměť) je klíčový prvek v moderních počítačích, který řeší obrovský nepoměr mezi rychlostí CPU a rychlostí operační paměti (RAM). Zatímco se rychlost procesorů dramaticky zvýšila (zdvojnásobení zhruba každých 18 měsíců, známé jako Mooreův zákon), rychlost RAM se tomu nevyrovnala. Cache je malá, ultra-rychlá paměť umístěná uvnitř CPU, která ukládá data nedávno přístupná nebo data, u nichž se očekává, že budou brzy potřeba. Tím se procesoru zkracuje cesta pro data, která by jinak musel složitě a pomaleji čerpat z RAM. Moderní CPU nepoužívají jen jednu úroveň cache, ale hierarchii: L1, L2 a L3

\begitems
* {\bf L1 Cache}: Nejrychlejší a nejmenší. Cache L1 je obvykle rozdělená na L1i (instrukce) a L1d (data), což odráží zjištění, že oddělené cache fungují lépe, protože oblasti paměti pro kód a data jsou do značné míry nezávislé. Je umístěna přímo uvnitř každého CPU jádra a každé jádro má svou vlastní.
* {\bf L2 Cache}: Větší, pomalejší než L1. Může být dedikovaná nebo sdílená mezi malou skupinou jader. Ukládá často přístupná data, která se nevešla do L1
* {\bf L3 Cache}: Největší a nejsdílenější. Na rozdíl od L1 (a často i L2) je sdílena mezi všemi jádry v rámci jednoho CPU
\enditems

CPU nepracuje s daty po jednotlivých bajtech, ale v blocích zvaných cache lines (řádky cache). Cache line je základní jednotka přenosu mezi pamětí a cache, přičemž standardní velikost je typicky 64 bajtů (ale závisí na architektuře procesoru).
\begitems 
* {\bf Cache Hit}: Když CPU potřebuje data, nejprve se podívá do L1 cache. Pokud jsou data nalezena (tzv. cache hit), jsou získána téměř okamžitě (během několika nanosekund).
* {\bf Cache Miss}: Pokud data nejsou v L1, CPU se postupně podívá do L2 a L3. Pokud data nejsou nalezena v žádné úrovni cache (cache miss), musí CPU přistoupit až do RAM. Přístup k RAM je 100krát pomalejší než přístup do L1 cache.
\enditems

\secc Prefetching (Přednačítání) 
CPU automaticky předpovídá, jaká data bude program potřebovat, a načítá je do cache dříve, než jsou explicitně vyžádána. {\bf Prefetching} funguje nejlépe, když program přistupuje k datům sekvenčně (např. procházení pole), protože takový přístup je předvídatelný. Při náhodném nebo nepředvídatelném přístupu (např. pointer chasing ve spojovém seznamu) prefetching selhává a program platí plnou cenu latence.

\secc Koherence Cache (Cache Coherency)
V systémech s více procesory (SMP) musí být zajištěno, že všechny procesory vidí stejný obsah paměti. Tato údržba jednotného pohledu na paměť se nazývá {\bf koherence cache}. CPU sledují (snoopují) operace zápisu prováděné ostatními jádry, aby zajistily, že pokud jedno jádro změní data, kopie v cache ostatních jader se označí jako neplatné (Invalid). K řízení tohoto procesu se používají protokoly jako {\bf MESI}.

\secc Falešné sdílení (False Sharing)
Závažný antivzor v multi-vláknovém programování, kdy dvě vlákna modifikují logicky nezávislé proměnné, které se ale nacházejí ve stejné cache line. Hardware považuje celou cache line za jednotku koherence, a tak se jádra neustále přetahují o vlastnictví dané cache line, což vede k neustálé invalidaci a zpomalení výkonu.


